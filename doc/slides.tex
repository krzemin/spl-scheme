\documentclass[12pt,serif]{beamer}
\usepackage[T1]{fontenc}
\usepackage{pxfonts}
\usepackage[utf8]{inputenc}
\usepackage[MeX]{polski}
\usecolortheme[RGB={20,80,150}]{structure} 
\setbeamercovered{invisible}
\usepackage{graphicx}


\usepackage{minted}
\usepackage{semantic}
\usepackage{syntax}

\title{SPL-Scheme}
\author{Piotr Krzemiński}
\date{Wrocław, 11 lutego 2014}

\begin{document}

\begin{frame}
\titlepage
\end{frame}


\begin{frame}{Motywacja i cele}
\begin{itemize}
  \item napisać trochę większy projekt w Haskellu (z~wykorzystaniem Cabala)
  \item zgłębić tajniki semantyki denotacyjnej, szczególnie w wersji CPS
  \item wykorzystać w praktyce wiedzę zdobytą na SJP
  \item m.in.: gorliwa ewaluacja, dynamiczne typowanie, kontynuacje, call/cc
\end{itemize}
\end{frame}

\begin{frame}{Dlaczego Scheme?}
\begin{itemize}
  \item niebardzo chcieliśmy skupiać się na pisaniu skomplikowanego
        parsera czy typecheckera
  \item dobra okazja, aby głębiej poznać rodzinę języków lispowych
\end{itemize}
\end{frame}

\begin{frame}{Scheme jest ustandaryzowany}
\begin{itemize}
  \item standard organizacji IEEE (The IEEE standard, 1178-1990 (R1995))
  \item raport $R^6RS$ (Revised6 Report on the Algorithmic Language Scheme)
\end{itemize}
\end{frame}

\begin{frame}{Nasze podejście}
\begin{center}
Zacząć od bardzo małego podzbioru, sukcesywnie dodając
nowe konstrukcje do języka
\end{center}
\end{frame}

\begin{frame}{Stan obecny}
\begin{itemize}
  \item parser (w Parsecu), pretty printer
  \item ewaluacja wyrażeń arytmetycznych, logicznych, sterujących
  \item lambda-abstrakcja i aplikacja (również wieloargumentowa)
  \item lispowe struktury danych (\texttt{cons}, \texttt{car}, \texttt{cdr}, \ldots)
  \item statycznie wiązane definicje zmiennych i funkcji
  \item rekursja
  \item konsola interaktywna (REPL)
  \item obsługa modułów ładowanych z plików
  \item interpreter
\end{itemize}
\end{frame}

\begin{frame}{Szczegóły implementacyjne}
\begin{itemize}
  \item semantyka denotacyjna w stylu kontynuacyjnym
  \item bardzo prosty core-language, dużo cukru syntaktycznego
        (np. jednoargumentowe lambdy w semantyce,
        wieloargumentowe definicje funkcji i~aplikacje jako cukier)
  \item definicje typów języka jako ADT w Haskellu
  \item typ wartości jako monada, kontynuacje jako prawy argument bind-a,
        dzięki czemu możemy w naturalny sposób używać do-notacji podczas
        implementowania semantyki
\end{itemize}
\end{frame}

\begin{frame}{Składnia}
\setlength{\grammarindent}{8em}
\begin{grammar}
<Exp> ::= $n$
\alt \#t | \#f
\alt atom
\alt "string"
\alt \textbf{(} <$ExpList$> \textbf{)}

<ExpList> ::= $\epsilon$
\alt <$Exp$> <$ExpList$>
\end{grammar}
\end{frame}

\begin{frame}{Dziedzina wartości}
$Env = Ide -> Val$ \\
$Cont = Env -> Val -> Val^{*}$ \\
$Clo = Val -> Cont -> Val^{*}$ \\
$Val = Exp \cup Clo$ \\
$Val^{*} \approx ((Env \times Val)~+~(\{err\} \times \Sigma^{*})~+~(\{typerr\} \times \Sigma^{*}))_{\bot}$\newline
\\
\pause
W implementacji:
\begin{itemize}
  \item $Val = Exp$
  \item $Exp~::=~\ldots~|~Clo$
\end{itemize}
\end{frame}


\begin{frame}{Dynamiczne typowanie}
$\iota_{num} : Exp -> Val$ \\
$\iota_{bool} : Exp -> Val$ \\
$\iota_{str} : Exp -> Val$ \\
$\iota_{list} : Exp^{*} -> Val$ \\
$\iota_{cons} : Val \times Val -> Val$ \\
$\iota_{clo} : Clo -> Val$ \\
\end{frame}

\begin{frame}[fragile]{Dynamiczne typowanie}
\footnotesize \begin{minted}[mathescape]{haskell}
data TypeDef repr = TypeDef {
  name :: String,
  toRepr :: Expr -> Maybe repr,
  fromRepr :: repr -> Expr
}

numType :: TypeDef Int
numType = TypeDef "num" extract Num where
  extract (Num n) = Just n
  extract _ = Nothing

boolType :: TypeDef Bool
boolType = TypeDef "bool" extract Bool where
  extract (Bool b) = Just b
  extract _ = Nothing
  
-- atom, string, cons, closure
\end{minted}
\end{frame}

\begin{frame}[fragile]{Dynamiczne typowanie}
\footnotesize \begin{minted}[mathescape]{haskell}
typed :: TypeDef repr -> (repr -> Val) -> Cont
typed t cont _ val =
  case (toRepr t) val of
    Just v -> cont v
    Nothing -> TypeErr ("Expected type `" ++ name t)

evalExpr (List [Atom "+", e0, e1]) env k =
  evalExpr e0 env $ typed numType $ \n0 ->
  evalExpr e1 env $ typed numType $ \n1 ->
  k env (Num (n0 + n1))
\end{minted}
\end{frame}


\begin{frame}{Funkcja semantyczna}
$|[ \cdot |]: Exp -> Env -> Cont -> Val^{*}$ \newline
\\
$|[n|] \eta \kappa$ = $\kappa$ $\eta$ $(\iota_{num}~n)$ \\
$|[b|] \eta \kappa$ = $\kappa$ $\eta$ $(\iota_{bool}$ $b$ $)$ \\
$|[x|] \eta \kappa$ = $\kappa$ $\eta$ $(\eta~x)$ \\
$|[s|] \eta \kappa$ = $\kappa$ $\eta$ $(\iota_{str}$ $s$ $)$ \newline
\\
$|[(\oplus~e_0~e_1)|] \eta \kappa$ = \\
\hspace{1em}{$|[e_0|]$ $\eta$ ($\lambda \eta_0 n_0$ .} \\
\hspace{2em}{$|[e_1|]$ $\eta$ ($\lambda \eta_1 n_1$ .} \\
\hspace{3em}{$\kappa$ $\eta$ ($\iota_{num}$ ($n_0 \oplus n_1)$)} \\
\hspace{2em}{$)_{num^{*}}$} \\
\hspace{1em}{$)_{num^{*}}$} \newline
\\
$\oplus \in \{+,-,*\}$
\end{frame}

\begin{frame}{Funkcja semantyczna}
$|[ \cdot |]: Exp -> Env -> Cont -> Val^{*}$ \newline
\\
$|[(\oslash~e_0~e_1)|] \eta \kappa$ = \\
\hspace{1em}{$|[e_0|]$ $\eta$ ($\lambda \eta_0 n_0$ .} \\
\hspace{2em}{$|[e_1|]$ $\eta$ ($\lambda \eta_1 n_1$ .} \\
\hspace{3em}{cond($n_1=0$, <err,"div. by 0">, $\kappa$ $\eta$ ($\iota_{num}$ ($n_0 \oslash n_1)$))} \\
\hspace{2em}{$)_{num^{*}}$} \\
\hspace{1em}{$)_{num^{*}}$} \newline
\\
$\oslash \in \{/,\%\}$ \newline
\\
\pause
Podobnie dla pozostałych operatorów (and, or, =, <, $\leq$, ...)
\end{frame}

\begin{frame}{Funkcja semantyczna}
$|[ \cdot |]: Exp -> Env -> Cont -> Val^{*}$ \newline
\\
$|[$(not $e$)$|] \eta \kappa$ =
  $|[e|]$ $\eta$ ($\lambda \eta' b$ . $\kappa$ $\eta$ $(\iota_{bool} (\neg b)))_{bool^{*}}$ \\
$|[$(cond $e$ $e_0$ $e_1$)$|] \eta \kappa$ = \\
\hspace{1em}{$|[e|]$ $\eta$ ($\lambda \eta' b$ . cond(b, $|[e_0|] \eta \kappa$, $|[e_1|] \eta \kappa$) $)_{bool^{*}}$} \newline
\\
$|[$(cons $e_0~e_1$)$|] \eta \kappa$ = \\
\hspace{1em}{$|[e_0|]$ $\eta$ ($\lambda \eta_0 v_0$ .} \\
\hspace{2em}{$|[e_1|]$ $\eta$ ($\lambda \eta_1 v_1$ .} \\
\hspace{3em}{$\kappa$ $\eta$ ($\iota_{cons}$ <$v_0, v_1$>)} \\
\hspace{2em}{)} \\
\hspace{1em}{)} \newline
\\
$|[$(car $e$)$|] \eta \kappa$ =
  $|[e|]$ $\eta$ ($\lambda \eta' $<$v1, v2$> . $\kappa$ $\eta$ $v1)_{cons^{*}}$ \\
$|[$(cdr $e$)$|] \eta \kappa$ =
  $|[e|]$ $\eta$ ($\lambda \eta' $<$v1, v2$> . $\kappa$ $\eta$ $v2)_{cons^{*}}$ \\
\end{frame}

\begin{frame}{Funkcja semantyczna}
$|[ \cdot |]: Exp -> Env -> Cont -> Val^{*}$ \newline
\\
$|[$(quote ($e_0 \ldots e_n$))$|] \eta \kappa$ =
  $\kappa$ $\eta$ $(\iota_{list} (e_0 \cdots e_n)))$ \newline
\\
$|[$(lambda $x$ $e$)$|] \eta \kappa$ =
  $\kappa$ $\eta$ $(\iota_{clo} (\lambda v \kappa'~.~ |[e|]~\eta[x \mapsto v]~\kappa' ))$ \newline
\\
$|[(e_0~e_1)|] \eta \kappa$ =
  $|[e_0|]$ $\eta$ ($\lambda \eta_0 f$ .
  $|[e_1|]$ $\eta$ ($\lambda \eta_1 v$ . $f~v~\kappa))_{clo^{*}}$ \newline
\\
$|[(e_1~e_2~e_3 \ldots e_n)|] \eta \kappa$ =
  $|[ (\ldots(e_0~e_1)~e_2) \ldots e_n) |]$ $\eta$ $\kappa$ \newline
\\
$|[$(letrec $f~x~e'~e$)$|] \eta \kappa$ = $|[e|]~\eta[f \mapsto fix~F]~\kappa$ \\
\hspace{1em}{gdzie:} \\
\hspace{2em}{ $F~g~v~\kappa' = |[e'|] ~ \eta[f \mapsto g][x \mapsto v] ~ \kappa' $ } \\
\end{frame}

\begin{frame}{Cukier syntaktyczny}
$\mathcal{D} : Exp -> Exp$ \newline
\\
$\mathcal{D}|[$ true $|]$ = \#t \\
$\mathcal{D}|[$ false $|]$ = \#f \newline
\\
$\mathcal{D}|[$ (if $b$ $e_0$ $e_1$) $|]$ =
   (cond $\mathcal{D}|[ b |]$ $\mathcal{D}|[ e_0 |]$ $\mathcal{D}|[ e_1 |]$) \newline
\\
$\mathcal{D}|[$ (let $x$ $e'$ $e$) $|]$ =
   ((lambda $x$ $\mathcal{D}|[e|]$) $\mathcal{D}|[e'|]$) \\
$\mathcal{D}|[$ (let* (($x_1$ $e_1$) .. ($x_n$ $e_n$)) $e$) $|]$ = 
   $\mathcal{D}|[$(let $x_1$ $e_1$ (..(let $x_n$ $e_n$ $e$)..))$|]$ \\
\end{frame}

\begin{frame}{Cukier syntaktyczny}
$\mathcal{D} : Exp -> Exp$ \newline
\\
$\mathcal{D}|[$ nil $|]$ = (quote ()) \\
$\mathcal{D}|[$ (nil? $e$) $|]$ = 
   (equals? $\mathcal{D}|[$ nil $|]$ $\mathcal{D}|[$ e $|]$) \newline
\\
$\mathcal{D}|[$ (lambda ($x_1 \ldots x_n$) $e$) $|]$ = \\
\hspace{2em}{$\mathcal{D}|[$(lambda $x_1$ (\ldots(lambda $x_n$ $\mathcal{D}|[e|]$)\ldots))$|]$}\newline
\\
$\mathcal{D}|[$ (defun ($f$ $x_1 \ldots x_n$) $e$) $|]$ = \\ 
\hspace{2em}{(define $f$ $\mathcal{D}|[$ (lambda ($x_1 \ldots x_n$) $e$ ) $|]$ )}\\

\end{frame}


\begin{frame}
\begin{center}
   \LARGE{Demo}
\end{center}
\end{frame}

\begin{frame}{Czego brakuje?}
\begin{itemize}
   \item rekursja wzajemna
   \item kontynuacje jako wartości pierwszego rzędu (call/cc)
   \item uboga biblioteka standardowa
   \item pokaźniejszy zbiór przykładowych programów testowych
   \item dokument z formalnym opisem semantyki
\end{itemize}
\end{frame}

\begin{frame}{Materiały}
\begin{itemize}
   \item Reynolds
   \item Racket
   \item Write yourself Scheme in 48 hours
   \item Notatki z SJP
\end{itemize}
\end{frame}

\end{document}
